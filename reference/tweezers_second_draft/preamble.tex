% Conveniently, the citeautoscript option in the revtex4-2 class toggles the spacing & punctuation automatically for superscript vs. bracketed citations. Place citations as if they were in [].
% Uncomment the line below to do superscript citations.
% \setcitestyle{super}

\usepackage{amsmath,amssymb} % math symbols
\usepackage{bm} % bold math font
\usepackage{graphicx} % for figures
\usepackage{comment} % allows block comments
\usepackage[normalem]{ulem} % allows strikeout text, e.g. \sout{text}

% \usepackage{minted} % allows colored code
% \usepackage{textcomp} % This package gives the text quote '

\usepackage{enumitem}
\setlist{noitemsep,leftmargin=*,topsep=0pt,parsep=0pt}

\usepackage{xcolor} % \textcolor{red}{text} will be red for notes
\definecolor{lightgray}{gray}{0.6}
\definecolor{medgray}{gray}{0.4}
\definecolor{mRed}{RGB}{230, 0, 50}
\colorlet{newtextColor}{mRed}

\usepackage{hyperref}
\hypersetup{
colorlinks=true,
urlcolor= blue,
citecolor=blue,
linkcolor= blue,
}

% Code to add paragraph numbers and titles
\newif\ifptitle
\newif\ifpnumber
\newcounter{para}
\newcommand\ptitle[1]{\par\refstepcounter{para}
{\ifpnumber{\noindent\textcolor{lightgray}{\textbf{\thepara}}\indent}\fi}
{\ifptitle{\textbf{[{#1}]}}\fi}}
%\ptitletrue  % comment this line to hide paragraph titles
%\pnumbertrue  % comment this line to hide paragraph numbers

% Code for reviewer text
%\newcommand{\revtext}[1]{\textcolor{reviewColor}{#1}}
\newcommand{\revtext}[1]{\textit{#1}}

% Code to track changes
\newif\iftrackchanges
\newcommand{\newtext}[1]
    {\textcolor{\iftrackchanges newtextColor\else black\fi}{#1}}
\newcommand{\deltext}[1]
    {\iftrackchanges{\textcolor{newtextColor}{\sout{#1}}}\fi}
%\trackchangestrue  % comment to hide tracked changes

% Instead of making TONS of colored newtext, let's just put a colored line next to big blocks of new text.
\usepackage{mdframed}
\newmdenv[
  linecolor={\iftrackchanges newtextColor\else white\fi},
  linewidth=2pt,
  topline=false,
  bottomline=false,
  rightline=false,
  skipabove=\topsep,
  skipbelow=\topsep,
  leftmargin=-12pt,
  innertopmargin=0pt,
  innerbottommargin=0pt
]{newtextblock}

% Uncomment this line if you prefer your vectors to appear as bold letters.
% By default they will appear with arrows over them.
% \renewcommand{\vec}[1]{\bm{#1}}

% Command to mark text in blue/red and define common units
\newcommand{\blue}{\textcolor{blue}}
\newcommand{\red}[1]{\textcolor{red}{#1}}
\newcommand{\df}{\dfrac}
\newcommand{\mic}{\mu\mathrm{m}}
\newcommand{\nm}{\mathrm{nm}}
\newcommand{\mm}{\mathrm{mm}}
\newcommand{\mrm}[1]{\mathrm{#1}}
\newcommand{\la}{\langle}
\newcommand{\ra}{\rangle}
\newcommand{\pd}[1]{\partial_{#1}}
% Command for labeled equations with custom alignment
\newcommand{\leqalign}[2]{
    \begin{equation}
        \begin{aligned}
            #2
        \end{aligned}
        \label{#1}
    \end{equation}
}
